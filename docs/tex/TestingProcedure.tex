\section{Testing Procedure}
% Per-requirement testing plan
% Processes, measurements, experiments
% NOTE: Often, acceptable ranges in results depend on specifics of the technology

\subsection{Acceptability}
% Are people willing to eat output? Incl. appearance, aroma, flavor, and texture
% Specific preparation and/or incorporation of output (formulation) for tasting

% Suggestion: Double-blind volunteer study
Tested via blind studies where participants are divided into two groups and given either control outputs (i.e. established commercial product) or test outputs (i.e. produced by PeaPod). Participants will rate outputs on 4 criteria (appearance, aroma, flavour, and texture) on a 9-point scale.

To simulate acceptability over a long period of time, it will be important to study outputs with consideration for how the subjects will interact with them. This includes varying preparation methods (fresh, cooked, dehydrated, etc.) and preparing combinations of foods both purely with PeaPod outputs and with external foods that would be available in the field.

Blind studies are eminent in consumer testing as they allow researchers to get a completely unbiased dataset. Special care needs to be taken when presenting, preparing, and collecting samples for testing to ensure researchers do not influence results. Ideally, resources will permit a double-blind study where researchers hire an outside entity to conduct the test and return results with generic labels.

The 9-point scale originates from U.S. Army testing, where it was developed using language which has roughly equal psychological distances between points on the scale \href{https://www.sensorysociety.org/knowledge/sspwiki/Pages/The%209-point%20Hedonic%20Scale.aspx}{source}. While the use of 9 points is otherwise arbitrary, there exists a large history of research validating its analytical use in long-term food production.

\subsection{Safety of Process}
% SPECIFICALLY food production area
% Chemical hazards - Toxins, heavy metals (As, Cd, Hg, Pb, etc.)
% Biohazards - total aerobic plate count, ATP testing of food contact surfaces
% TODO: What do those^ mean? How do we test them?

% NOT materials, NOT the system itself
% Cleaning, disinfection (chemical selection and frequency), harvesting/crew interaction procedures
% !! Mitigation of threats both pre-flight (sterilization and packaging) and in the field (minimal testing, minimal crew interaction)



\subsubsection{Chemical Hazards}
As per the parts manifest !LINK, all components involved in the unit are clear of toxins and heavy metals by composition.

% All static materials, inputs (nutrients, pH)

\subsubsection{Biological Hazards}
Aerobic Plate Count (APC) testing to be done with the Conventional Plate Count Method outlined in the FDA's \href{https://www.fda.gov/food/laboratory-methods-food/bam-chapter-3-aerobic-plate-count}{BAM Chapter 3: Aerobic Plate Count}. This is selected over the Spiral Plate Method as it is inexpensive and uses many household materials. The goal of APC testing is to indicate the bacterial population in food-adjacent sections of the design. Results to be compared against STD-3001 to ensure a maximum of 3000 colony forming units/square ft. Plate count to be minimized by following !CITE (surface cleaning standards? hard to find). 

ATP testing to be done using !CITE (lots of stuff about methods but no standards? look at requirements more)

% Seeds (seed cleaning and disinfection, packaging)

\subsection{Safety of Outputs}
% SPECIFICALLY food outputs
% Biohazards - total aerobic count, specific pathogens (enterobacteriaceae, salmonella, yeasts, molds, E. coli, Listeria, etc.)

% So long as all threats are mitigated for both process and materials/inputs, food outputs should be safe

APC testing conducted on samples as outlined above to ensure bacterial population below 20 000 CFU/g per STD-3001. Testing for enterobacteriaceae will be performed using the MicroSnap EB rapid test to ensure its population is below 100 CFU/g per STD-3001. Testing for salmonella will be performed using a rapid detection kit to ensure a population of 0 CFU/g.
 Testing for yeasts and molds will be performed with a testing kit to then be analyzed for a population count below 1000 CFU/g.

Critical pathogens to be tested for individually:
%most testing procedures are a product (like MicroSnap)--is this in budget?
\begin{itemize}
    \item Enterobacteriaceae: 100 CFU/g
    \item Salmonella: 0 CFU/g
    \item Yeast and Molds: 1000 CFU/g
    \item Escherichia Coli: dep. on tech
    \item Listeria: dep. on tech
\end{itemize}


\subsection{Resource Outputs}
% Nutritional analysis - macro- and micro-nutrients
% TODO: How to perform nutritional analysis?
% Really depends on plant selection. How do we know what the nutritional makeup of a given food product is at the time of harvest?
Personally testing nutritional makeup of outputs is far beyond the resources and scope of this project. Instead, a variety of outputs will be produced and shipped to an external, ISO-17025 certified lab such as SGS Canada for testing.

% There are no inherent micro- or macro-nutritional limitations imposed by the system and its food outputs.

\subsection{Reliability and Stability of Outputs}
% Biohazards - total aerobic plate count, enterobacteriaceae, yeasts, molds
%is this not the same as safety of outputs?

% Fresh produce is inherently non-shelf-stable
% What preservation methods are available to us (refrigeration?)
% Staggering planting times so that harvests happen at regular intervals

PeaPod's outputs are intended for consumption as soon as possible after harvest and preparation. Any product not immediately consumed post preparation should be stored in an airtight package and kept below 12.5°C to slow bacterial growth. Certain products of PeaPod have the possibility of being dehydrated to further increase their shelf life. 
To mitigate the need for long duration food storage, growth cycles should be staggered to periodically supply fresh produce when astronauts are ready to eat.(Look into dehydrating processes and shi)

\section{Sample Collection Procedure and Schedule}
% Days/cycles of operation before sample collection?
% Incl. sample collection (size, timing, quantity), packaging, shipping
The number of days required for sample collection is entirely dependent on what sample is being produced. For one-time growth products, such as carrots or lettuce, the days required is exactly the time to harvest of the plant. Size of collection is dependent on how many units are run at the same time. For plants that produce products multiple times, such as beans or tomatoes, samples should be collected after each production cycle. This means the time required to collect n samples is C + n*X, where C is the initial growth period of the plant and X the time between harvests. It is important to collect multiple subsequent harvests in order to see the relationship between this and produce quality.

Packaging and shipping will be done according to freight standards of the carrier being used, such as \href{https://www.fedex.com/en-us/shipping/how-to-ship-perishables.html#1}{this guide from FedEx}. 

% Sample collection procedure?
% Collect non-edible plant matter (roots, stems, leaves, flowers, etc.) for safety testing