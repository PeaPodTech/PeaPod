\textbf{Control Range and Parameter Independence}

Unlike large-scale vertical farming, PeaPod’s isolation and parameter independence lets it simulate any climate. Wide LED spectrum can emit both near-infrared and near-ultraviolet light, important for creating hormonal responses and compounds in plants. Combined with insulation, humidification and dehumidification, and thermoelectric heating/cooling, PeaPod can generate extreme environments and even conditions on other planets (minus gravity and atmospheric changes). 

These parameters are independent: e.g., lighting heat is countered by thermoregulation cooling. Also, inline nutrient and pH solution injection eliminates drawbacks of reservoir use by taking less space, less solution mixing time, and avoiding a control loop of nutrients/pH (while also being more accurate). 

%These innovations allow PeaPod to grow virtually any plant. 

\textbf{Form Factor and Extendability}

The range of output environments is also possible because of the form factor of a single PeaPod unit. Warehouse-scale vertical farming cannot provide wide-spectrum control due to size---poor insulation and air circulation prohibits extreme temperatures. PeaPod solves this with a small, modular design that enables robust lighting, heating, and cooling systems at home and at scale.

Space savings are a benefit of this feature, as the output of an entire farming or hydroponics setup (requiring a flat field or warehouse) can be spread through unused space (corners, under shelves/desks, etc.) via many small PeaPod setups. This means a large yield can be had without construction, zoning, labour, or any of the other issues accompanying large farming or hydroponic setups.

\textbf{Optimization}

Existing approaches to plant optimization are simple and ineffective, relying on a \textbf{fixed/unchanging} environment parameter set and only examining \textbf{final} plant metrics. This approach is severely limited, in that it does not account for changes over time.

Instead, statistical model is used which takes into account the cumulative property of growth. By monitoring all environment and plant indicators, repeatable and controlled trials with scientific validity are able to be performed. PeaPod counters declining health/quality indicators in real time, and generates tailored programs that to maximize any metric and target further improvement.

% See the preliminary calculations Appendix \ref{app:optimization} for details.

\textbf{Open-Source Design and Data}

Since PeaPod is standard and open-source, units can be had in bulk and assembled by anybody. Public contribution to the project (both in design improvement and data) increases the reliability and safety of the solution. Collected data is, ideally, committed to the public, meaning anybody with a PeaPod can run the same iteration with the same program and species to boost scientific validity, or run a different program to expand PeaPod's knowledge base.