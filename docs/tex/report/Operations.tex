Once PeaPod has been assembled, the first step is manual sterilization to prevent microbial growth at high growth environment humidities. Any plant- and food-safe means may be employed (i.e. UV, wipe-down with disinfectant followed by distilled water).

Next is the hookup of power, network, and water reservoir inputs, as well as the filling of the water reservoir (reverse osmosis, nutrient-rich and pH-balanced for crop choice).

PeaPod's operation begins at planting. Seeds are removed from vacuum-sealed storage bags and distributed evenly on mesh media, which is then moistened for germination and placed in the grow tray. The front panel is then closed, sealing the internal environment. Once sprouted, plants may be harvested as microgreens or transplanted to neoprene pucks inside grow cups for further growth.

The desired state of the growth environment is encoded via a standardized set of "environment parameters". The specific values of each parameter for each iteration are stored in a "program". The program is set prior to the start of plant growth, and consists of a set of \textbf{actions} (e.g. set red LEDs to 63\% power) and \textbf{control targets} (e.g. hold air temperature at 22°C) as time-series instructions from the start of planting to the point of harvest (or plant death, in the case of multiple-harvest plants) corresponding to the control systems.

When activated, PeaPod's control systems begin to enact the program. PeaPod will notify crew of any required action, including refilling inputs, cleaning components, and harvesting, as well as "End-of-Program", when all inedible plant matter is disposed of. The process can then be repeated (starting at sterilization).