Being a sustainable isolated unit, PeaPod requires little cleaning, limited to UV sterilization each time it is opened.

PeaPod uses safe materials in its chassis, insulation and circuitry. 
The main frame is constructed using aluminum. Although large quantities of aluminum in food are deemed dangerous, the lack of direct contact exposure of aluminum to the plants passes well below the toxicity limit \cite{aluminum}.
The bracketing and mounts of PeaPod are constructed using PETG plastic which has been deemed “food-safe plastic” \cite{petg}.

The foam insulation used in PeaPod is commonly used for housing and other common safe applications. 

To avoid toxins in circuitry, lead-free soldering was used for all electronics. The dehumidification system uses silica gel, which is commonly found in food packets and is deemed “biodegradable and non-toxic” \cite{silica}.

All voltages are sub 48V DC, avoiding any high-voltage risks. The voltage risk is also mitigated by short-circuit/overcurrent protection. 
All pressures experienced by the aeroponics system stay below 100 PSI, avoiding dangers with high pressures. The dangers with pressures are also mitigated through the use of PTFE tape, fail-safe solenoids (which primarily stay closed) and a pressure sensor shutoff. 

Materials selection eliminates the risk of off-gassing. The presence of microbes or other harmful pathogens are mitigated through the use of clean seeds, reverse osmosis water and pure nutrient/pH solutions, as well as "gas exchange equilibrium" and UV sterilization pre- and post-opening. HEPA filters on both gas exchange ports and the dehumidification cartridge (which is removed from the chamber) mitigate microbial and aerosol presence. Through a nutrient injection manifold, PeaPod also has the ability to administer custom solutions, such as anti-pathogenic compounds (fungicides, algicides).
In addition, PeaPod provides plant nutrients directly without the use of nitrogen-fixing bacteria. The production process is fully automated, mitigating all risks associated with human error. 

In the event of a malfunction, PeaPod also allows the user to override the program for the purposes of editing or shutting down the unit.