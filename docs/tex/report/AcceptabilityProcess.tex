\textit{Footprint}:

Due to PeaPod's modular construction, the footprint can vary. For a Standard Extended PeaPod, the footprint would be 2m x 0.5m, or about three standing refrigerators. When stowed, volume is reduced to 37\% of when assembled. Control module is packed pre-assembled.

The following estimates are for a Standard Extended PeaPod.

\textit{Setup Process}:

\begin{enumerate}
    \item Assemble housing - \textbf{2 hours}
    \item Install control module(s):
    \begin{enumerate}
        \item Hook up water, power, and network inputs - \textbf{5 min}
        \item Fill nutrient and pH adjustment solution containers - \textbf{10 min}
        \item Mount CM to housing - \textbf{5 min}
    \end{enumerate}
    \item Assemble all trays - \textbf{1 hour},
    \item For each tray, either:
    \begin{enumerate}
        \item Mount lighting boards and driver, daisy chain boards to driver, hook up power and signal to driver and CM - \textbf{10 min} per tray, \textbf{OR}
        \item Mount aeroponic nozzle mount and arm, hook up water delivery line to nozzles and CM - \textbf{10 min} per tray
    \end{enumerate}
    \item UV sterilization - \textbf{20 min}
    \item Prepare and plant seeds for desired crop output, seal growth environment - \textbf{30 min}
    \item Enable primary power supply, and power on automation system, allow to perform self-test and calibrations - \textbf{10 min}
    \item Open water input shutoff valve
    \item Input program for required environments and activate - \textbf{5 min}
\end{enumerate}

Total setup process time (Standard: 2 trays per unit, 12 units, 1 CM): \textbf{8.5 hours} (\textit{one person}) or \textbf{2.5 hours} (\textit{crew of 4})

\newpage

\textit{Food Production Cycle}:

\begin{enumerate}
    \item Environment is maintained, and environment can be observed live at a computer terminal via sensor data and camera feed
    \item Circulation fans enable automated pollination
    \item Perform maintenance, including:
    \begin{itemize}
        \item Cleaning nozzle once a month - \textbf{10 min}
        \item Swapping and recharging dehumidification cartridges when instructed - \textbf{5 min} (active time)
        \item Refilling solution containers when instructed - \textbf{5 min}
    \end{itemize}
    \item Upon End-Of-Program (EOP) notification, the gas exchange system will conduct a "full equalization flush", bringing the internal environment in equilibrium with the surroundings. Users will harvest and store food products (or prepare and consume them immediately, varying time) - \textbf{15 min}
    \item All organic material remaining in the system is removed and disposed of, presence of residue or algae is determined - \textbf{10 min}
    \item If algae present, hydrogen peroxide is used to kill the growth - \textbf{5 min}
    \item (Optional) In event of significant contamination, system can be flushed with an all-purpose surface disenfectant and then rinsed with distilled water - \textbf{10 min}
    \item Upon End-Of-Life notification (may occur at the same time as EOP), the plant is scrapped (\textbf{15 min}), and new plants may be planted
\end{enumerate}

Total maintenance time per week: \textbf{1-2 hours} (depending on program)

\textit{Process Evaluation}:

Setup and maintenance processes are fully documented in a "User Manual", which includes both text instructions (with numerical specifications for different actions) as well as diagrams for reference. Notifications from computer refer users to specific subsections of the Manual for maintenance actions. All processes require no specific expertise, just the ability to operate basic hand tools and follow instructions.

All interactions with the automation system (i.e. program upload, environment/camera monitoring) can be accomplished either via a touchscreen panel on the front of the control module or over the Internet.