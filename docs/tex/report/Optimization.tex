\textbf{High Success Rates}: Complete automation and environmental control ensures high crop success rates and yield predictability.

\textbf{Repeatability}: Once optimal conditions are found for a given crop species, they can be repeated ad infinitum.

\textbf{Immediate Sensor Feedback and Response}: Immediate feedback from both environment sensors and plant metric analysis empowers the system to respond to unpredictable or otherwise uncontrolled factors (i.e. poor seed health, outside interference). Plant metric analysis can be used to diagnose program ineffectualities, accelerate optimization, and preventatively mitigate plant health decline.

\textbf{Data Collection, Yield Optimization}: By collecting data via computer vision and post-harvest yield evaluation (GCMS, weighing, etc.) on the plant's response to the induced environment, the relationship between the species behaviour and the surrounding environment can be analyzed. Plant metrics include plant health indicators (chlorophyll concentrations/chlorosis, leaf count/size distribution/density, plant height/canopy dimensions leaf tip burn, leaf curl, wilting, etc.) and crop yield (edible matter net mass/percent mass of plant, total plant mass, chemical/nutritional composition, caloric measurement, etc.). Data is filtered/smoothed across time to account for noise. The relationship is then represented by a statistical/machine learning model via a method known as "surrogate modelling". The method for this analysis can be found in the preliminary calculations Appendix \ref{app:optimization}.