\section{Testing Plan}
% Per-requirement testing plan
% Processes, measurements, experiments
% NOTE: Often, acceptable ranges in results depend on specifics of the technology

\subsection{Acceptability of Outputs}
% Are people willing to eat output? Incl. appearance, aroma, flavor, and texture
% Specific preparation and/or incorporation of output (formulation) for tasting

\subsubsection{Testing Procedure}
% How will this requirement be tested within the context of your specific technology?
% Suggestion: Double-blind volunteer study
Tested via blind studies where participants are divided into two groups and given either control outputs (i.e. established commercial product) or test outputs (i.e. produced by PeaPod). Participants will rate outputs on 4 criteria (appearance, aroma, flavour, and texture) on a 9-point scale.

To simulate acceptability over a long period of time, it will be important to study outputs with consideration for how the subjects will interact with them. This includes varying preparation methods (fresh, cooked, dehydrated, etc.) and preparing combinations of foods both purely with PeaPod outputs and with external foods that would be available in the field.

Blind studies are eminent in consumer testing as they allow researchers to get a completely unbiased dataset. Special care needs to be taken when presenting, preparing, and collecting samples for testing to ensure researchers do not influence results. Ideally, resources will permit a double-blind study where researchers hire an outside entity to conduct the test and return results with generic labels.

The 9-point scale originates from U.S. Army testing, where it was developed using language which has roughly equal psychological distances between points on the scale. While the use of 9 points is otherwise arbitrary, there exists a large history of research validating its analytical use in long-term food production \cite{hedonic}.

\textbf{1382 Characters} (Maximum 4000)

\subsubsection{Sample Collection Schedule}
% How many days/ cycles will the system be in operation before samples can be collected, to clearly assess the above criteria? Please provide a description of the process for sample collection (including size, timing and number of samples), packaging, and shipping.

%to address acceptability, will need a variety of things at a variety of qualities based on real distributions

To test acceptability of a long-term food solution, it is important to test the entire spectrum of output quality at a good-faith approximation of its distribution. This means instead of testing 'good' control outputs against 'good' PeaPod outputs, we will need to test batches of control outputs (at a naturally-arising quality distribution) vs. batches of PeaPod outputs (again, at a quality distribution). This allows for holistic comparison of the technology's performance as well as end user experience as opposed to specific comparison of individual items.

To employ statistical analysis, a sample size of at least $n=30$ will be collected for each crop, with the collection time dependent on the crop. For initial tests, crops will be collected at a developmental stage roughly equivalent to control crops purchased at a store. For further tests, the crops can be collected at the point PeaPod's analytical tools determine is optimal. Packaging and shipping will be done according to existing best practices, with care to package more than necessary as a factor of safety. 

\textbf{1079 Characters} (Maximum 4000)

\clearpage

\subsection{Safety of Process and Outputs}
% SPECIFICALLY food production area
% Chemical hazards - Toxins, heavy metals (As, Cd, Hg, Pb, etc.)
% Biohazards - total aerobic plate count, ATP testing of food contact surfaces
% TODO: What do those^ mean? How do we test them?

% NOT materials, NOT the system itself
% Cleaning, disinfection (chemical selection and frequency), harvesting/crew interaction procedures
% !! Mitigation of threats both pre-flight (sterilization and packaging) and in the field (minimal testing, minimal crew interaction)

\subsubsection{Testing Procedure}
% How will this requirement be tested within the context of your specific technology?

Given the environment in which PeaPod will operate, process safety will be developed on the foundation of prevention. This is because crisis response and containment is severely limited in the confines of space: identification often requires propagation of the threat (i.e. incubation of potential pathogens for colony count), and quarantine is more difficult and loses a larger proportion of food than it does on Earth.

This begins pre-flight, as all materials - especially biological - are sanitized, tested, and packaged in isolation so a breach contaminates as little product as possible. Once everything is installed in the field, the design principles of the entire system take over as methods of prevention. By employing key design principles such as minimal interaction, PeaPod mitigates the introduction of foreign substances and, in turn, the ingress of potential threats. Interaction will only occur at times of harvest and planting, which double as times of cleaning and sanitation. 

For testing the UX of the processes, using established space station procedures, subjects will harvest and clean product, clean all surfaces, sanitize all surfaces, and plant surface-sterilized seeds.

As for non-biological threats such as heavy metals or other toxins, careful selection and sourcing of construction materials will eliminate most threats to the system. As a regular maintenance measure, flushing of the water and air supplies through the space station's recycling system will prevent buildup by keeping them up to external standards.

\textbf{Chemical Hazards}\\
As part of PeaPod's design principle regarding prevention, all sourcing of parts and resources has been done with inherent, chemical threats in mind (lead-free solder and electronic components, food-safe fittings and parts for aeroponics, etc.). As a result, the default construction of the unit poses no threat for chemicals or other toxins to enter the biological system or its surroundings.

The other source of potential threats is during crew interaction steps when they harvest, clean, sanitize, and plant. To protect against hazards, PeaPod's maintenance steps follow carefully designed HACCP protocols and use food-safe cleansers and sanitizers.

% All static materials, inputs (nutrients, pH)

\textbf{Biological Hazards}\\
There are no biological hazards inherent to the PeaPod system. Plant species selection should be performed carefully to ensure safety according to mission requirements.

Aerobic Plate Count (APC) testing to be done with the Conventional Plate Count Method outlined by the FDA \cite{platecount}. This is selected over the Spiral Plate Method as it is inexpensive and uses many household materials. The goal of APC testing is to indicate the bacterial population in food-adjacent sections of the design. Results to be compared against STD-3001 to ensure a maximum of 3000 colony forming units/square ft. Colony forming units appear as distinct ``blobs'' of bacteria on a growth material, indicating the relative abundance of viable bacteria on a given surface.

\textbf{Food Outputs}
% Biohazards - total aerobic count, specific pathogens (enterobacteriaceae, salmonella, yeasts, molds, E. coli, Listeria, etc.)

% So long as all threats are mitigated for both process and materials/inputs, food outputs should be safe
APC testing conducted on samples as outlined for biological hazards to ensure bacterial population below 20 000 CFU/g per STD-3001. \textit{Enterobacteriaceae} have a population limit of 100 CFU/g per STD-3001, and can be tested via a rapid test such as the MicroSnap EB. Similarly, testing for salmonella will be performed using a rapid detection kit to ensure a population of 0 CFU/g. Finally, testing for yeasts and molds will be performed with a testing kit to to ensure a population count below 1000 CFU/g.

Given the wide use cases for crops (raw/dried, cut/whole, fresh/preserved) it will be important to conduct tests on each use case to see if results remain acceptable. For example, higher surface area crops are intrinsically more susceptible to bacterial growth per gram. So, crops prior to testing will be processed in the same way they would be before consumption in order to validate realistic operating conditions.
%microsnap source: https://www.hygiena.com/food-safety-solutions/indicator-organisms/microsnap-enterobacteriaceae/

% Seeds (seed cleaning and disinfection, packaging)

\textbf{By-Product Outputs}
By-product outputs fall into two cateogries, air exhaust and inedible matter. Air will be filtered and dehumidified to ambient levels by the gas exchange system, ensuring it is safe for the local environment. Inedible matter will be tested in the same way as food outputs to ensure threats are not present in the system at all. Further threats would arise in the processing of by-products should they be used as a nutrient source in successive cycles. If this process is implemented, new testing procedures will be developed for it. Otherwise, threats are mitigated by disposal in the same manner as other biological substances such as human waste.

% maximum 4000 characters

\textbf{4036 Characters} (Maximum 4000)

\subsubsection{Sample Collection Schedule}
% How many days/ cycles will the system be in operation before samples can be collected, to clearly assess the above criteria? Please provide a description of the process for sample collection (including size, timing and number of samples), packaging, and shipping.
Safety testing will be conducted in tandem with collection of samples for acceptability, with key metrics being measured as successive rounds of crops are collected. Beyond number and size of samples, it will be critical to follow operating procedures as closely as possible. This is because of its two-fold interaction with crop quality and safety. The only deviation will be the measuring of metrics such as CFU count, as this is safe to do on Earth conditions.

Regardless of test results, crop collection will continue to see if the design's mitigation strategies are effective. For example, if Harvest 2 has allowable but present levels of \textit{Enterobacteriaceae}, it is important to see if it persists to Harvests 3 and 4 or if standard cleaning procedures manage to eliminate it.

Then, over the course of 30 or so harvests, the presence of hazards can be charted as a function of harvest number and cleaning cycles, allowing patterns to emerge and weaknesses to be found for long-duration missions. Testing for these hazards will be in line with the list above: bacteria via APC, organic residue via ATP testing, and chemical hazards via water supply analysis. Water supply testing is straightforward, with an abundance of commercial test strips available to identify any given substance.

\textbf{1172 Characters} (Maximum 4000)

\clearpage

\subsection{Resource Outputs}
% Nutritional analysis - macro- and micro-nutrients
% TODO: How to perform nutritional analysis?
% Really depends on plant selection. How do we know what the nutritional makeup of a given food product is at the time of harvest?

\subsubsection{Testing Procedure}
% How will this requirement be tested within the context of your specific technology?

Personally testing nutritional makeup of outputs is far beyond the resources and scope of this project. Instead, a variety of outputs will be produced and shipped to an external, ISO-17025 certified lab such as SGS Canada for testing.

While this is useful for validation on Earth, it fails to address the issue of analysis at time of harvest. To tackle this, PeaPod will use data collected on Earth-bound trials in combination with lab analysis and existing datasets to develop a way of predicting crop quality during the growing process. By applying algorthmic prediction, we can optimize resource output efficiency by, for example, marking crops that show early signs of failure for replacement. This helps maximize the output to input ratio by cutting losses earlier and with less labour cost than people would be able to.

% There are no inherent micro- or macro-nutritional limitations imposed by the system and its food outputs.

\textbf{826 Characters} (Maximum 4000)

\subsubsection{Sample Collection Schedule}
% How many days/ cycles will the system be in operation before samples can be collected, to clearly assess the above criteria? Please provide a description of the process for sample collection (including size, timing and number of samples), packaging, and shipping.

The number of days required for sample collection is entirely dependent on what sample is being produced. For one-time growth products, such as carrots or lettuce, the days required is exactly the time to harvest of the plant. Size of collection is dependent on how many units are run at the same time. For plants that produce products multiple times, such as beans or tomatoes, samples should be collected after each production cycle. This means the time required to collect n samples is C + n*X, where C is the initial growth period of the plant and X the time between harvests. It is important to collect multiple subsequent harvests in order to see the relationship between this and produce quality.

Packaging and shipping will be done according to freight standards of the carrier being used, such as \cite{shipping}. 

% Sample collection procedure?
% Collect non-edible plant matter (roots, stems, leaves, flowers, etc.) for safety testing

\textbf{824 Characters} (Maximum 4000)

\clearpage

\subsection{Reliability and Stability of Outputs}
% Biohazards - total aerobic plate count, enterobacteriaceae, yeasts, molds
%is this not the same as safety of outputs?

% Fresh produce is inherently non-shelf-stable
% What preservation methods are available to us (refrigeration?)
% Staggering planting times so that harvests happen at regular intervals

\subsubsection{Testing Procedure}
% How will this requirement be tested within the context of your specific technology?

PeaPod's outputs are fresh produce, and as such are intended for consumption as soon as possible after harvest and preparation. Any product not immediately consumed post preparation should be stored in an airtight package and kept below 12.5°C to prolong shelf life and maintain acceptability. Certain products of PeaPod have the possibility of being dehydrated to further increase their shelf life.

To mitigate the need for long duration food storage, growth cycles should be staggered to periodically supply fresh produce when astronauts are ready to eat.

\textbf{558 Characters} (Maximum 4000)

\subsubsection{Sample Collection Schedule}
% How many days/ cycles will the system be in operation before samples can be collected, to clearly assess the above criteria? Please provide a description of the process for sample collection (including size, timing and number of samples), packaging, and shipping.

Collection time is not a critical variable for this test, however safe storage time is. To quantify it, a variety of outputs can be compared to their control equivalents to identify any disparities in shelf life---be it raw, dehydrated and cooled, or airtight and cooled.

The number of cycles necessary to complete testing will be equivalent to the number of items tested. Each of these three storage methods should be tested for a variety of plant species and yield types over a number of trials to minimize randomness.

Criteria for shelf life will include acceptability as a function of time, quantity/existence of bacteria, and the relationship between crew time and quality of output.

\textbf{690 Characters} (Maximum 4000)

\clearpage

\subsection{Additional Comments}
% Days/cycles of operation before sample collection?
% Incl. sample collection (size, timing, quantity), packaging, shipping

% Maximum 4000 characters

\clearpage

\subsection{Materials}
\subsubsection{System}

\textbf{Automation}\\

\begin{table}[!h]
    \centering
    \begin{tabular}{|c|l|l|l|c|}
    \hline
        Index   & Manufacturer Part Number  & Manufacturer Name         & Description                       & Quantity  \\ \hline
        1       & A000005                   & Arduino                   & ARDUINO NANO ATMEGA328 EVAL BRD   & 1         \\ \hline
        2       & S404GSEJ6-U3000-3         & Delkin Devices, Inc.      & 4GB MLC MICROSD CARD (-25C - +85  & 1         \\ \hline
        3       & 61304021121               & Würth Elektronik          & CONN HEADER VERT 40POS 2.54MM     & 1         \\ \hline
        4       & SC0510                    & Raspberry Pi              & ZERO 2 W                          & 1         \\ \hline
        5       & DMN2005K-7                & Diodes Incorporated       & MOSFET N-CH 20V 300MA SOT23-3     & 2         \\ \hline
        6       & RC0603FR-0710KL           & YAGEO                     & RES 10K OHM 1\% 1/10W 0603        & 5         \\ \hline
        7       & 4484                      & Adafruit Industries LLC   & MINI PITFT 1.3 FOR RASPBERRY PI   & 1         \\ \hline
        8       & 5055670271                & Molex                     & CONN HEADER SMD R/A 2POS 1.25MM   & 2         \\ \hline
        9       & 5055670471                & Molex                     & CONN HEADER SMD R/A 4POS 1.25MM   & 5         \\ \hline
        10      & 5055670871                & Molex                     & CONN HEADER SMD R/A 8POS 1.25MM   & 3         \\ \hline
        11      & 5055670681                & Molex                     & CONN HEADER SMD R/A 6POS 1.25MM   & 3         \\ \hline
    \end{tabular}
    \caption{Automation system electronic components.}
    \label{tab:automation_components}
\end{table}

In addition, 1x \textit{Automation Motherboard PCB}: 2 Layers, 1 oz. Copper, 1.6mm Thickness, Suggested: HASL Finish (Lead-Free), White PCB, Black Silkscreen

\textbf{Housing}\\

\begin{table}[!ht]
    \centering
    \begin{tabular}{|l|l|c|l|l|}
    \hline
        Part                    & Description                                           & Quantity  & Supplier          & Supplier Part Number  \\ \hline
        Control Module Housing  & 5-Sided enclosure                                     & 1         & Protocase         & ~                     \\ \hline
        Frame Front X Extrusion & Silver Painted, 20x20mm, Ordered 2ft., Cut to 500mm   & 2         & McMaster-Carr     & 5537T101              \\ \hline
        Frame Door Y Extrusion  & Silver Painted, 20x20mm, Ordered 2ft., Cut to 500mm   & 2         & McMaster-Carr     & 5537T101              \\ \hline
        Frame Rear X Extrusion  & Silver Painted, 20x20mm, Ordered 2ft., Cut to 460mm   & 2         & McMaster-Carr     & 5537T101              \\ \hline
        Frame Door X Extrusion  & Silver Painted, 20x20mm, Ordered 2ft., Cut to 460mm   & 2         & McMaster-Carr     & 5537T101              \\ \hline
        Frame Rear Y Extrusion  & Silver Painted, 20x40mm, Ordered 2ft., Cut to 460mm   & 2         & McMaster-Carr     & 5537T111              \\ \hline
        Frame Front Y           & Silver Painted, 20x20mm, Ordered 2ft., Cut to 460mm   & 2         & McMaster-Carr     & 5537T101              \\ \hline
        Frame Z                 & Silver Painted, 20x20mm, Ordered 2ft., Cut to 460mm   & 4         & McMaster-Carr     & 5537T101              \\ \hline
        Tray X Extrusion        & Silver Painted, 20x20mm, Ordered 2ft., Cut to 440mm   & 4         & McMaster-Carr     & 5537T101              \\ \hline
        Tray Z Extrusion        & Silver Painted, 20x20mm, Ordered 2ft., Cut to 400mm   & 6         & McMaster-Carr     & 5537T101              \\ \hline
        Nozzle Arm Extrusion    & Silver Painted, 20x20mm, Ordered 1ft., Cut to 150mm   & 2         & McMaster-Carr     & 5537T101              \\ \hline
        M5x0.8 10mm Bolts       & Alloy Steel, Black Oxide Coated, Socket Head Cap      & 139       & McMaster-Carr     & 91290A224             \\ \hline
        M5x0.8 16mm Bolts       & Alloy Steel, Black Oxide Coated, Socket Head Cap      & 10        & McMaster-Carr     & 91290A232             \\ \hline
        M4x0.7 16mm Bolts       & Alloy Steel, Black Oxide Coated, Socket Head Cap      & 12        & McMaster-Carr     & 91290A154             \\ \hline
        M4x0.7 Hex Nuts         & High-Strength Steel, Black Oxide Coated               & 12        & McMaster-Carr     & 94166A110             \\ \hline
        M5 T-Nuts               & Zinc-Plated Steel, Black Painted, for 5mm Slot        & 139       & McMaster-Carr     & 5537T651              \\ \hline
        M5x0.8 Hex Nuts         & Alloy Steel, Black Oxide Coated                       & 10        & McMaster-Carr     & 94166A120             \\ \hline
        Foam Insulation         & DUROSPAN GPS R5 4ft. x 8ft. x 1in.                    & 1         & Home Depot Canada & 1001211234            \\ \hline
        Reflective Mylar        & 27in. x 12ft. x 0.002in., Aluminum-coated PET         & 1         & McMaster-Carr     & 7538T11               \\ \hline
        Adhesive                & LePage PL300 Foamboard 295mL                          & 2         & Home Depot Canada & 1000403469            \\ \hline
        Grow Cup                & 2in. Diameter                                         & 16        & Amazon            & ~                     \\ \hline
        Door Hinges             & Plastic, Black, for 20x20mm Extrusion                 & 2         & McMaster-Carr     & 5537T85               \\ \hline
        Feet Bumpers            & Adhesive-Back, Medium-Hard Polyurethane, Black        & 4         & McMaster-Carr     & 95495K24              \\ \hline
    \end{tabular}
    \caption{Housing subsystem components.}
    \label{tab:housing_parts}
\end{table}

\begin{table}[!ht]
    \centering
    \begin{tabular}{|l|c|l|l|}
    \hline
        Part                        & Quantity  & Materials             & Process           \\ \hline
        L Bracket                   & 12        & PETG Filament         & 3D Printing       \\ \hline
        L Bracket (Grow Tray)       & 4         & PETG Filament         & 3D Printing       \\ \hline
        Diagonal Bracket            & 4         & PETG Filament         & 3D Printing       \\ \hline
        T Bracket                   & 10        & PETG Filament         & 3D Printing       \\ \hline
        Tray Hook BL                & 2         & PETG Filament         & 3D Printing       \\ \hline
        Tray Hook BR                & 2         & PETG Filament         & 3D Printing       \\ \hline
        Tray Hook FL                & 2         & PETG Filament         & 3D Printing       \\ \hline
        Tray Hook FR                & 2         & PETG Filament         & 3D Printing       \\ \hline
        Grow Plate Quarters         & 4         & 210x210x5mm PET Sheet & Table Saw         \\ \hline
        Grow Plate Washer           & 1         & PETG Filament         & 3D Printing       \\ \hline
        Nozzle Mount A              & 2         & PETG Filament         & 3D Printing       \\ \hline
        Nozzle Mount B              & 2         & PETG Filament         & 3D Printing       \\ \hline
        Lighting LED Board Mount    & 5         & PETG Filament         & 3D Printing       \\ \hline
        Lighting Power Board Mount  & 1         & PETG Filament         & 3D Printing       \\ \hline
        Feet                        & 4         & PETG Filament         & 3D Printing       \\ \hline
    \end{tabular}
    \caption{Housing subsystem fabricated parts.}
    \label{tab:housing_fabrication}
\end{table}

\textbf{Aeroponics}\\


\textbf{Leaf-Zone Thermoregulation}\\


\textbf{Humidification}\\


\textbf{Dehumidification}\\


\textbf{Gas Composition Regulation and Exchange}\\


\textbf{Lighting}\\


\subsubsection{Inputs}
% Supply inputs (water, power, network), consumable inputs (pH/nutrient solutions, dehumidification cartridge)

\textbf{Supply Inputs}
\begin{itemize}
    \item \textit{Water}: reverse-osmosis, ambient
    \item \textit{Power}: 120V 60Hz AC\footnote{The power supply can be altered to suit a variety of power inputs (i.e. DC)}
    \item \textit{Network}: ethernet or wireless, optional
\end{itemize}

\textbf{Consumable Inputs}
\begin{itemize}
    \item \textit{Nutrient/pH Adjusment Solutions}: pouches
    \item \textit{Dehumidification Cartridge}: recharged
\end{itemize}

\subsubsection{Outputs}
% Food outputs(?), by-products/waste (waste water from flushing, dehumidification cartridge)

\textbf{Food Outputs}\\


\textbf{By-Products \& Waste}\\


\subsubsection{Maintenance}

\textbf{Spare Components}\\


\textbf{Tools}\\


\subsubsection{Cleaning}

\textbf{Soaps}\\


\textbf{Disinfectants}\\


\textbf{Tools}\\

