\documentclass{report}
\usepackage{setspace} % Setting line spacing
\usepackage{ulem} % Underline
\usepackage{caption} % Captioning figures
\usepackage{subcaption} % Subfigures
\usepackage{geometry} % Page layout
\usepackage{multicol} % Columned pages
\usepackage{array,etoolbox}
\usepackage{fancyhdr}
\usepackage{enumitem}
\usepackage[toc,page]{appendix}
\setlist{noitemsep}

% Page layout (margins, size, line spacing)
\geometry{letterpaper, left=1in, right=1in, bottom=1in, top=1in}
\setstretch{1}

% Headers
\pagestyle{fancy}
\lhead{PeaPod - Testing Plan}
\rhead{PeaPod Technologies Inc.}

\begin{document}

\begin{titlepage}
    \begin{center}
        \vspace*{1.2cm}

        \textbf{\large{PeaPod - Testing Plan}}

        \vspace{0.5cm}

        NASA/CSA Deep Space Food Challenge Phase 2

        \vfill
        \small{
    \textbf{Jayden Lefebvre - Founder, Lead Engineer}\\
    Port Hope, ON, Canada\\
    \vspace{.5cm}
    \textbf{Nathan Chareunsouk - Design Lead}\\Toronto, ON, Canada\\
    \vspace{.5cm}
    \textbf{Navin Vanderwert - Design Engineer}\\
    BASc Engineering Science (Anticipated 2024), University of Toronto, Toronto, ON, Canada\\
    \vspace{.5cm}
    \textbf{Jonas Marshall - Electronics Engineer}\\
    BASc Computer Engineering (Anticipated 2024), Queen's University, Kingston, ON, Canada\\
    \vspace{.5cm}
    Open-source contributions made by:\\
    \textbf{University of Toronto Agritech}
}

\vspace{1cm}

Primary Contact Email: contact@peapodtech.com
        \vspace{.75cm}

        Revision 0.1\\
        PeaPod Technologies Inc.\\
        January 9th, 2022

    \end{center}
\end{titlepage}

\thispagestyle{plain}

\tableofcontents
\newpage

\section{Testing Procedure}
% Per-requirement testing plan
% Processes, measurements, experiments
% NOTE: Often, acceptable ranges in results depend on specifics of the technology

\subsection{Acceptability}
% Are people willing to eat output? Incl. appearance, aroma, flavor, and texture
% Specific preparation and/or incorporation of output (formulation) for tasting

% Suggestion: Double-blind volunteer study?

\subsection{Safety of Process}
% SPECIFICALLY food production area
% Chemical hazards - Toxins, heavy metals (As, Cd, Hg, Pb, etc.)
% Biohazards - total aerobic plate count, ATP testing of food contact surfaces
% TODO: What do those^ mean? How do we test them?

\subsection{Safety of Outputs}
% SPECIFICALLY food outputs
% Biohazards - total aerobic count, specific pathogens (enterobacteriaceae, salmonella, yeasts, molds, E. coli, Listeria, etc.)

\subsection{Resource Outputs}
% Nutritional analysis - macro- and micro-nutrients

\subsection{Reliability and Stability of Outputs}
% Biohazards - total aerobic plate count, enterobacteriaceae, yeasts, molds

\section{Sample Collection Procedure and Schedule}
% Days/cycles of operation before sample collection?
% Incl. sample collection (size, timing, quantity), packaging, shipping

\section{Hazard Analysis and Critical Control Point (HACCP) Plan}
% Def'n of terms: https://www.fda.gov/food/hazard-analysis-critical-control-point-haccp/haccp-principles-application-guidelines#princ
% HACCP reference: https://spinoff.nasa.gov/moon-landing-food-safety?utm_source=TWITTER&utm_medium=KathyLueders&utm_campaign=NASASocial&linkId=141839394

\subsection{Food Production System Description}
% Incl. flowchart

\subsection{Critical Points}

\subsubsection{Critical Point A}
% Hazard description
% Critical limits
% Monitoring procedures
% Deviation procedures
% Associated documents

\subsubsection{Critical Point ...}

\subsection{Standard Test Record}

\subsubsection{Purpose and Summary}

\subsubsection{Safety and Quality}

\subsubsection{Test Processes}

% Preparation of inputs
% Verification
% Setup, maintenance, and collection protocols
% Storage
% Cleanup and Turnover

\subsubsection{Closeout}

\newpage

% References
\bibliographystyle{IEEEtran}
\bibliography{references}
\end{document}