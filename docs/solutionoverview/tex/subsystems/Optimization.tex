\subsection{Optimization}
\label{sec:optimization}

\textbf{Purpose}: Continuously improve yield/etc. of crops as more environment parameter and crop metric data is gathered across iterations.

\textbf{Method}: 

Assume a plant's growth rate (or state change) is related to its current internal state $\vec P \in \R^n$ (for $n$ plant metrics) and the environment conditions $\vec E \in \R^m$ (for $m$ environment parameters). Let these both be functions $\vec P (t),\vec E(t)$ defined at each $t$, where $t=0$ indicates the time of planting. Assume that this relationship is constant for all members of a given species.

Define plant state change $\vec P'$: 

$$\vec P'(t) = \frac{d}{dt}\vec P(t)$$

Define the plant-environment behaviour function $Q$: 

$$Q(\vec P(t), \vec E(t), t)=\vec P'(t)$$ 

Given the current internal and external states, determine the plant's state change.

\begin{enumerate}
    \item Set $\vec E_{set}(t)~\forall~ t$, aka the program (\ref{sec:automation});
    \item Record $\vec P(t)~\forall~ t$ and $\vec E(t)\approx \vec E_{set}(t)~\forall~ t$ (\ref{sec:environment});
    \item Calculate $\vec P'(t)~\forall~ t$;
    \item Fit $\vec Q$ to our data;
\end{enumerate}

By fitting $\vec Q$ across iterations, we can predict $\vec P$ at any $\vec E$ and $t$. For example:

$$\vec P(t+\Delta t)=P(t)+\Delta t\cdot Q(\vec P(t),\vec E(t))$$

\textbf{Features}:
\begin{itemize}
    \item \textit{Machine Learning Model}: Represents $Q$. Operates on the main automation hardware, or in the cloud.
    \item \textit{Environment Sensors}: See Section \ref{sec:automation}. Collects $\vec E$ over time.
    \item \textit{Plant Metrics}: See Section \ref{sec:automation}. Collects $\vec P$ over time.
\end{itemize}