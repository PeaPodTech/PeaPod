\subsection{Automation}
\label{sec:automation}

\textbf{Purpose}: Performing growth-, maintenance-, and data-related tasks autonomously on the basis of both schedule and necessity to reduce crew maintenance time. Maintains the homogeneity of the internal environment. Increased accuracy/precision over human interference, minimize human hours spent. Enables control over all parameters simultaneously.

\textbf{Function}:
\begin{itemize}
    \item \textbf{Inputs}: Environment sensor reading signals, program
    \item \textbf{Outputs}: Actuator control signals, crew messaging
\end{itemize}

\textbf{Method}:
\begin{enumerate}
    \item \textit{Setup}:
    \begin{enumerate}
        \item Power is connected and system is booted;
        \item Program is inputted by user;
    \end{enumerate}
    \item \textit{Testing}:
    \begin{itemize}
        \item Power-on Self-Test (POST) passes;
        \item Systems enact program as intended;
    \end{itemize}
    \item \textit{Process}:
    \begin{enumerate}
        \item Checks operating preconditions (self POST and per-subsystem);
        \item \textbf{Environment Control Loop} (matches \textit{Sense-Plan-Act} model of robotics):
        \begin{enumerate}
            \item Receives and stores data about current environment state;
            \item Compares current state to desired state, develops a "plan" to reach desired state;
            \item Controls subsystem operations in order to enact the plan;
        \end{enumerate}
        \item Notifies user on maintenance requirement (i.e. non-automated input/output management, refills, repairs, etc.) and end-of-program (EOP);
    \end{enumerate}
    \item \textit{Shutdown} (either manual or EOP):
    \begin{enumerate}
        \item Stop subsystem operations;
    \end{enumerate}
\end{enumerate}

\textbf{Features}:
\begin{itemize}
    \item \textit{Computer System}: Manages \textbf{all} data collection, storage, analysis, and transmission/receiving, as well as planning and actuator control. Includes internal clock (for program, notification), network connection (for data transmission, notification), and storage (for data). 
    \item \textit{Camera \& Plant Metrics}: Multiple angles. For live feed transmission to users (local and remote), as well as plant health and yield metric collection via \textbf{computer vision analysis}. Matches $\vec P$ from the optimization routine (\ref{sec:optimization}). Metrics include:
    \begin{itemize}
        \item Leaf health indicators (i.e. leaf tip burn, leaf curl, chlorosis);
        \item Leaf count, size distribution;
        \item Leaf density;
        \item Canopy dimensions/surface area;
        \item Plant height;
        \item Fruit/harvest body size, ripeness;
        \item etc.
    \end{itemize}
    \item \textit{Environment Sensors}: Record the environment's current state. Covers each environment control loop (see \ref{sec:environment} \textbf{outputs}). Matches $\vec E$ from the optimization routine (\ref{sec:optimization}).
\newpage
    \item \textit{Diagnostic Systems}: Include informative sensors tracking system input availability, etc. as well as notification triggers.
    \item \textit{Program}: Set of action (e.g. lights on) and control target (e.g. hold air temperature at 22°C) \textbf{time-series} instructions;
    \item \textit{Actuator Control}: Induce a change. Covers each environment control (see \ref{sec:environment}, \ref{sec:lighting} \textbf{inputs});
\end{itemize}