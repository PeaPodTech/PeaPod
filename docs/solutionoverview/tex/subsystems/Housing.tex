\subsection{Housing}
\label{sec:housing}

\textbf{Purpose}: \textit{Isolates} and \textit{insulates} growth environment from surroundings (heat, light, water vapour, air). Provides structural integrity and mounting points for other subsystems. Enables system extendability via repeated "unit cell" topology.

\textbf{Method}:
\begin{enumerate}
    \item \textit{Setup}:
    \begin{enumerate}
        \item Construct frame and install panels;
        \item Mount control module (w/ subsystems), connect inputs and internal subsystem connections;
        \item Install tray mounts, insert trays (w/ subsystems);
    \end{enumerate}
    \item \textit{Testing}:
    \begin{itemize}
        \item Frame construction is rigid, level, and sturdy;
        \item Panels are insulating against temperature changes;
    \end{itemize}
    \item \textit{Process}:
    \begin{enumerate}
        \item Panels insulate against heat gain/loss, are opaque, and contain light and heat via reflection;
        \item Shell construction is tight, thus sealing against moisture;
        \item Internal vertical mounting channels for systems, horizontal plane "trays";
        \item \textbf{Extension} (can be repeated):
        \begin{enumerate}
            \item Add a second housing;
            \item Remove dividing panel from both housings;
            \item Remove "shared" skeleton extrusions from second housing;
            \item Join the two housings to form one larger 2x1 housing;
            \item \textbf{Extension Modes} (may be combined in any way to suit application):
            \begin{itemize}
                \item \textit{Option 1} (Smaller Housings): Operate the combined housing off \textbf{one} control module.
                \item \textit{Option 2} (Larger Housings): Add control modules to account for additional air volume, plant count, power requirement, etc.. Operate in a \textbf{controller-follower topology}.
                \item \textit{Option 3} (Frame Connection Only): Leave the dividing panel, add a control module, and operate the two PeaPods \textbf{separately}.
            \end{itemize}
        \end{enumerate}
    \end{enumerate}
    \item \textit{Shutdown}:
    \begin{enumerate}
        \item Dismount all systems, remove trays;
        \item Disassemble housing;
    \end{enumerate}
\end{enumerate}

\textbf{Features}:
\begin{itemize}
    \item \textit{Frame}: T-slotted aluminum extrusion framing with aluminum face-mounted brackets forms a cubic skeleton for rigidity/strength (high strength-to-weight aluminum) and easy component mounting and repositioning (standard mounting channels). These extrusions form the "edges" of the cubic housing. %Todo: cite extrusion
    \item \textit{Panels}: Graphite-enhanced expanded polystyrene (GPS) rigid foam insulation panels \cite{insulation} with reflective mylar internal lamination increase energy efficiency (GPS RSI of 0.0328$\frac{m^2 \cdot \degree C}{W}$ per mm of thickness, mylar enables light/heat reflection), as well as safety against cross-contamination and pathogens. Panels slide into extrusion channels and form a "seal" for greater water vapour retention. These panels form the "faces" of cube. %Todo: cite mylar
    \item \textit{Trays}: Horizontal plane subframes mounted to internal vertical extrusion channels for ease of repositioning. Trays slide in/out on permanent mounts. All connections are quick-connect (i.e. quick-diconnect tubing for grow tray, push connectors for lighting) for ease of removal. Trays include:
    \begin{itemize}
        \item \textit{Grow Trays}: Support plants (via grow cups), aeroponic nozzles, aeroponics container, and supply/recycling lines (See \ref{sec:aeroponics}).
        \item \textit{Lighting Trays}: Support LED boards, driver board (See \ref{sec:lighting}).
    \end{itemize}
\end{itemize}