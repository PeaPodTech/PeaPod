\subsection{Air Dehumidification}
\label{sec:dehum}

\textbf{Purpose}: Actively \textit{decreases} growth environment air humidity.

\textbf{Function}:
\begin{itemize}
    \item \textbf{Inputs}: Humid air (high water vapour content), dehumidification control signal (shutter servo open/close, fan speed), dry desiccant
    \item \textbf{Outputs}: Dry air (low water vapour content), saturated desiccant
\end{itemize}

\textbf{Method}:
\begin{enumerate}
    \item \textit{Setup}:
    \begin{enumerate}
        \item Dehumidification control signals are hooked up;
        \item Dry desiccant is added to cartridge, which is inserted;
    \end{enumerate}
    \item \textit{Testing}:
    \begin{itemize}
        \item Desiccant removes moisture from air;
        \item Desiccant indicates saturation as expected, which is sensed by computer;
        \item Shutters operate as intended, and no dehumidification occurs when closed;
        \item Maximum dehumidification rate exceeds total plant transpiration rate;
    \end{itemize}
    \item \textit{Process}:
    \begin{enumerate}
        \item Humidity sensor sends data to control module (\ref{sec:automation});
        \item Dehumidification control signal activates fans and opens shutters;
        \item Humid air passes over the desiccant, and dry air exits the unit;
        \item Desiccant becomes saturated, and indicates this;
        \item Indication is sensed by computer (\ref{sec:automation}), which notifies the user;
        \item Cartridge is removed by the user, and swapped for a dry one. Process continues;
        \item Saturated cartridge is recharged;
    \end{enumerate}
    \item \textit{Shutdown}:
    \begin{enumerate}
        \item Control signals are disconnected;
        \item Final recharging of cartridge;
        \item Desiccant is removed from cartridge;
    \end{enumerate}
\end{enumerate}

\textbf{Calculations}:

Assuming an air temperature of 30$\degree$ C, water vapour saturation of $30.4g/m^{3}$, RH of 90\%, target RH of 20\%, and 6\% weight dessicant capacity:
\vspace{.05cm}
\begin{gather*}
    90\% RH = 0.90 * 30.4g/m^{3} = 27.36g/m^{3} \\
    20\% RH = 0.20 * 30.4g/m^{3} = 6.08g/m^{3} \\
    Vol_{4 Units} = 0.5m * 0.5m * 0.5m * 4 = 0.5m^{3} \\
    Mass_{water} = (27.36g/m^{3} * 0.5m^{3}) - (6.08g/m^{3} * 0.5m^{3}) = 10.64g \\
    \frac{10.64}{0.06} = 177.3g \\
\end{gather*}

$\therefore$ 177.3g of dessicant will change the RH\% of a 4 unit setup from 90\% to 20\%.


\textbf{Features}:
\begin{itemize}
    \item \textit{Humidity Sensor}: See Section \ref{sec:airhum}.
    \item \textit{Dehumidification Unit}: One input port and one output port. Comprised of:
    \begin{itemize}
        \item \textit{Fans}: Draws moist air through input port and dried air through output port.
        \item \textit{Filter}: HEPA filter is located at inlet of dehumidification chamber. Eliminates risk of any airborne pathogens being transferred onto silica beads.
        \item \textit{Airtight Shutters} - Isolates dehumidification chamber when not in use. Prevents unintended dehumidification. Located at input and output ports. Controlled by a servo.
        \item \textit{Desiccant Cartridge}: Oven-safe. Easily removable for swapping and "recharging".
        \item \textit{Silica Beads}: Cheap, efficient, non-toxic desiccant. Changes color when saturated. Can be reused indefinitely when water is evaporated.
    \end{itemize}
    \item \textit{Evaporator Oven}: \textbf{Onboard systems} provide an oven that can maintain 200°C for 60 minutes. Heats cartridge to evaporate/"bake off" moisture collected by silica beads, thus "recharging" them. Vapour is collected by onboard dehumidifier.
\end{itemize}